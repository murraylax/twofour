\documentclass[11pt]{article}
\usepackage[T1]{fontenc}
\usepackage{calc}
\usepackage{setspace}
\usepackage{multicol}
\usepackage{fancyheadings}
\usepackage[round]{natbib}

\usepackage{graphicx}
\usepackage{color}
\usepackage{rotating}
\usepackage{verbatim}

\setlength{\voffset}{-0.3in}
\setlength{\topmargin}{0pt}
\setlength{\hoffset}{-0.3in}
\setlength{\oddsidemargin}{0pt}
\setlength{\headheight}{0pt}
\setlength{\headsep}{0in}
\setlength{\marginparsep}{0pt}
\setlength{\marginparwidth}{0pt}
\setlength{\marginparpush}{0pt}
\setlength{\footskip}{.3in}
\setlength{\textwidth}{7.4in}
\setlength{\textheight}{9.4in}
\setlength{\parskip}{0pc}
\setlength{\parindent}{2pc}

\renewcommand{\baselinestretch}{1.4}

\newcommand{\bi}{
  \begin{itemize}
  \setlength{\itemsep}{0pt}
  \setlength{\parindent}{0pt}  
  \setlength{\parskip}{0pt}
}
\newcommand{\ei}{\end{itemize}}
\newcommand{\be}{
  \begin{enumerate}
  \setlength{\itemsep}{0pt}
  \setlength{\parskip}{0pc}
}
\newcommand{\ee}{\end{enumerate}}
\newcommand{\bd}{\begin{description}}
\newcommand{\ed}{\end{description}}
\newcommand{\prbf}[1]{\textbf{#1}}
\newcommand{\prit}[1]{\textit{#1}}
\newcommand{\beq}{\begin{equation}}
\newcommand{\eeq}{\end{equation}}
\newcommand{\bdm}{\begin{displaymath}}
\newcommand{\edm}{\end{displaymath}}
\newcommand{\script}[1]{\begin{cal}#1\end{cal}}
\newcommand{\citee}[1]{\citet*{#1}}
\newcommand{\h}[1]{\hat{#1}}
\newcommand{\ds}{\displaystyle}

\newcommand{\toprule}{\par\vspace*{2pt}\noindent{\hrule\hfill}\par\vspace*{1pt}}


\newcommand{\app}
{
\appendix
}

\newcommand{\appsection}[1]
{
\let\oldthesection\thesection
\renewcommand{\thesection}{Appendix \oldthesection}
\section{#1}\let\thesection\oldthesection
\renewcommand{\theequation}{\thesection\arabic{equation}}
\setcounter{equation}{0}
}

\pagestyle{plain}

\begin{document}
\thispagestyle{empty}
\setcounter{page}{1}

\begin{center}\textbf{CBA FACULTY/ACADEMIC STAFF PROPOSAL FUND APPLICATION}\end{center} 

\noindent TITLE: Comparing the Learning Experiences in the First Two Years: Two Year vs Four Year Institutions\\

\ \vspace*{-2.5pc}

\noindent 1. Primary Applicant: James Murray, Department of Economics, 8 years service to UWL \\

\ \vspace*{-2.5pc}

\noindent 2. Budget request: TOTAL REQUEST: \$5,400 = Stipend: \$5,000 + Visa gift cards for survey participation incentive: 10 x \$40 = \$400.  The survey will take approximately 15-20 minutes to complete.  To encourage full participation from as many respondents as possible, we will give \$40 Visa gift cards to ten randomly chosen respondents from Western Technical College, Viterbo University, and the University of Wisconsin - La Crosse.\\

\ \vspace*{-2.5pc}

\noindent 3. Abstract: Our goal is to identify differences in students' learning experiences in two-year versus four-year institutions that may influence subsequent academic or career success.  We recognize that many students begin their college careers at two-year institutions with the intention of transferring to a four-year institution and earning a four-year baccalaureate degree.  We wish to identify whether students that transfer from two year programs are as prepared as others for their subsequent years in college in three dimensions: (1) mindset, (2) writing knowledge and confidence, and (3) information-seeking behavior.  For each goal we will produce a paper to submit for peer-reviewed publication. We have constructed a single electronic survey to be administered to first- and second-year students at UWL, Viterbo, and Western Technical College that measures mindset, writing ability and confidence, information-seeking behavior, prior academic experience, demographic information, and intentions for seeking baccalaureate or advanced degrees. While others have compared two-year and four-year students' academic success, the proposed work is innovative in terms of identifying differences in students' thought processes and behaviors, which is information that can inform specific improvements on advising, curriculum, and policy in higher education.\\

\ \vspace*{-2.5pc}

\noindent 4. Proposal Narrative

\noindent \underline{A. Objectives and Outcomes}: This project will result in submissions of \textbf{three separate manuscripts} to be submitted to \textbf{peer reviewed journals}.  The first submission will be in May 2017.

\noindent \textit{1. Student Mindset}: The first paper will measure differences between two-year and four-year students regarding their attitudes on their abilities to learn.  A person that has a fixed mindset for a particular skill views his or her abilities in that skill as relatively fixed.  \textbf{Manuscript submission date: May 2017} \\

\noindent \textit{2. Information-Seeking Behavior}: This paper is an extension of my previous work where I identify student information seeking behavior (knowledge, use, and comfort with library and academic resources for research).  We will measure and compare UWL, Viterbo, and WTC students' information seeking behavior, while controlling academic and demographic characteristics. \textbf{Manuscript submission date: August 2017} \\

\noindent \textit{3. Writing Ability and Confidence}: In the final paper we will measure and compare students' writing ability and confidence, focusing on differences in knowledge, experience, and confidence with writing concepts including audience, purpose, style, voice, genre, and grammar, while controlling academic and demographic characteristics.  \textbf{Manuscript submission date: October 2017} \\

\noindent \underline{B. Impact}: These papers will identify the relative strengths and weaknesses of two-year versus four-year students that can inform instructors, such as myself, administrators, and policy makers at both types of institutions on specific curricular improvements to improve students' chances for academic success.  We are considering the following journals: \textit{Higher Education Quarterly} (ABDC 2016 Rating: B), \textit{Journal of Higher Education} (ABDC 2016 Rating: B), \textit{Higher Education: The International Journal of Higher Education Research} (ABDC 2016 Rating: A), \textit{Studies in Higher Education} (ABDC 2016 Rating: A) (Source: \texttt{http://www.abdc.edu.au/master-journal-list.php}). \\

\noindent \underline{C. Project Time Commitment}:
The following is an estimate of the time commitment on work completed through June 1, 2017.  The survey has already been developed, approved by the UWL Institutional Review Board, and circulated to UWL students. This research project is a collaboration with Sloan Komissarov (Lecturer, Western Technical College), Brenda Murray (Associate Lecturer, University of Wisconsin - La Crosse), and Sara Cook (Professor, Viterbo University). We each bring unique expertise that makes the project possible, we will each carry an equal workload, and our names will appear in alphabetical order in all manuscript submissions.  The time commitment estimates below includes only the contributions made by James Murray. 

\noindent 1. Collect background information and literature related to mindset and write introduction and literature review for the mindset paper.  Time: 3 weeks, approximately 20 hours.\\
\noindent 2. Data cleaning and arranging and statistical analysis. Time: 2 weeks, approximately 15 hours.\\
\noindent 3. Write methodology and results sections for the mindset paper. Time: 1 week, approximately 10 hours.\\
\noindent 4. Write conclusions / discussions section for the mindset paper. Time: 1 week, approximately 10 hours.\\

\noindent \underline{D. Past Grants}

\noindent 1. CBA Research Grant 2013:  Title: ``Identifying Regime Switching in Adaptive Expectations.''  Presented the work at the 2016 Midwest Applied Time Series and Econometrics annual meeting.\\
2. UWL Faculty Research Grant 2012-2013.  Title: ``Labor Markets and Adaptive Expectations: Estimating a New Keynesian Model with Learning and Unemployment.'' This project evolved with a change in title and a focus (remaining true to original intent to estimate the impact that adaptive expectations on unemployment).  Work presented at the 2014 Midwest Economics Association annual conference and manuscript submitted to the \textit{Journal of Macroeconomics} in 2015.  \\
3. CBA Research Grant 2011.  Title: ``Fiscal and Expenditure Multipliers When There are Adaptive Expectations.''  Presented the work at the 2011 Midwest Economics Association annual conference.\\
4. UWL Faculty Research Grant 2010-2011.  Title: ``Academic Benefits of Living on Campus: A look at Peer Influences and Utilization of University Provided Resources'' Two publications in \textit{Economics Bulletin} (2010) and \textit{American Journal of Business Education} (2010).  
  

\end{document}



